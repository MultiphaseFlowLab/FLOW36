\chapter{Single phase flow}

\section{Continuity equation}
Hypothesis of uncompressible flow. $\mathbf u$ is the fluid velocity and it can be splitted in its three components: $u$, $v$ and $w$.
\begin{equation}
\nabla \cdot\mathbf u=\frac{\de u}{\de x}+\frac{\de v}{\de y} +\frac{\de w}{\de z}=0
\end{equation}

\section{Navier--Stokes equations}
Hypothesis of uncompressible flow.
\begin{equation}
\rho \frac{\de \mathbf u}{\de t} +\rho \mathbf u \cdot \nabla \mathbf u =-\nabla p +\mu \Nabla^2 \mathbf u
\end{equation}
We define $\Nabla^2$ as:
\[
\Nabla^2=\frac{\de^2}{\de x^2}+\frac{\de^2}{\de y^2}+\frac{\de^2}{\de z^2}
\]
The product $\mathbf u \cdot \nabla \mathbf u$ has to be intended as $(\mathbf u\cdot\nabla)\mathbf u$:
\renewcommand\arraystretch{1.5}
\[
\mathbf u \cdot \nabla \mathbf u=(\mathbf u\cdot\nabla)\mathbf u=\left( u\frac{\de}{\de x} +v\frac{\de}{\de y}+w\frac{\de}{\de z}\right) \mathbf u =
\begin{bmatrix}
u\frac{\de u}{\de x} +v\frac{\de u}{\de y}+w\frac{\de u}{\de z}\\
u\frac{\de v}{\de x} +v\frac{\de v}{\de y}+w\frac{\de v}{\de z}\\
u\frac{\de w}{\de x} +v\frac{\de w}{\de y}+w\frac{\de w}{\de z}\\
\end{bmatrix}
\]
\renewcommand\arraystretch{1}

\subsection{Nondimensionalization of Navier--Stokes equation}
We introduce the generic length- and velocity-scale of the flow, respectively $L_c$ and $u_c$, then we start the nondimensionalization using these two quantities. The superscript $*$ denotes nodimensional quantities.
\[
\frac{u_c}{\frac{L_c}{u_c}} \frac{\de \mathbf u^*}{\de t^*}+u_c \mathbf u^* \cdot \frac{1}{L_c}\nabla^* u_c \mathbf u^*=-\frac{1}{\rho}\frac{1}{L_c}\rho u_c^2\nabla^* p^*+\frac{\mu}{\rho}\frac{1}{L_c^2}\Nabla^{*2} u_c \mathbf u^*
\]
Reworking the eqation, we obtain:
\[
\frac{\de \mathbf u^*}{\de t^*}+\mathbf u^* \cdot \nabla^* \mathbf u^*=-\nabla^* p^*+\frac{\mu}{\rho u_cL_c}\Nabla^{*2} \mathbf u^*
\]
The dimensionless group $u_cL_c\rho/\mu$ is the Reynolds number:
\[
\Re=\frac{\rho u_c L_c}{\mu}
\]
The nondimensional Navier--Stokes equation results in:
\begin{equation}
\frac{\de \mathbf u^*}{\de t^*}+\mathbf u^* \cdot \nabla^* \mathbf u^*=-\nabla^* p^*+\frac{1}{\Re}\Nabla^{*2} \mathbf u^*
\end{equation}
From now on we will use only nondimensional quantities, unless differently specified, so for a more compact notation the superscript $*$ will be dropped.\\
At this point we introduce the domain used in the simulations (see Fig \ref{fig: domain_sketch}). Its size is $4\pi h\times2\pi h\times 2h$ in the $x$, $y$, $z$ directions\footnote{The $x$ and $y$ dimensions are chosen so that all the two-points correlation vanish from the inlet to the outlet.}; the main flow occurs in the $x$ direction. In the $x$ and $y$ directions are enforced periodic boundary conditions, while at $z=\mp1$ the boundary conditions depend on the case chosen (open channel, closed channel, ...).
\begin{figure}[H]
\centering
\def\svgwidth{0.8\textwidth}
\input{immagini/domain_sketch.pdf_tex}
\caption{Sketch of the domain}
\label{fig: domain_sketch}
\end{figure}
We define now the tho reference quantities previously stated: $L_c$ is the channel half-height $h$, while $u_c$ is the shear velocity $u_\tau$.
\[
\begin{array}{l}
L_c=h \\ [3px]
u_c=u_\tau=\sqrt{\dfrac{\tau_w}{\rho}}
\end{array}
\]
The value of the shear stress at the wall $\tau_w$ can be easily obtained from a force balance and depends on the geometry studied:
\begin{itemize}
\item Open channel:
\[
\tau_wL_xL_y=\Delta\bar p L_y 2h
\]
\[
\tau_w=2\frac{\Delta \bar p}{L_x}h
\]
\item Closed channel:
\[
2\tau_wL_xL_y=\Delta\bar p L_y 2h
\]
\[
\tau_w=\frac{\Delta \bar p}{L_x}h
\]
\end{itemize}
$\Delta \bar p$ is the time- and space-averaged pressure gradient in the flow direction $x$.\\
The pressure gradient can be splitted in two components: a mean one and a fluctuating one.
\[
\nabla p=\nabla \bar p+\nabla p'=\Pi+\nabla p'
\]
This way the values of the wall shear stress can be rewritten as $\tau_w=2\Pi h$ for the open channel case and as $\tau_w=\Pi h$ for the closed channel case.\\
In the code \texttt{FLOWSB} the mean pressure gradient in the $x$ direction is $\Pi=-1$.\\
The shear Reynolds number Re$_\tau$ is defined as:
\begin{equation}
\Re_\tau=\frac{\rho u_\tau h}{\mu}
\end{equation}
For the open channel the shear velocity is:
\begin{equation}
u_\tau^{oc}=\sqrt{\frac{2\Pi h}{\rho}}
\end{equation}
while for the closed channel it is:
\begin{equation}
u_\tau^{cc}=\sqrt{\frac{\Pi h}{\rho}}
\end{equation}
We recall the dimensionless Navier--Stokes equation (the superscript $*$ has been dropped for brevity and the pressure gradient has been divided in a mean one and in a fluctuating one):
\[
\frac{\de \mathbf u}{\de t}=-\mathbf u\cdot \nabla \mathbf u-\nabla p'-\Pi+\frac{1}{\Re_\tau}\Nabla^2 \mathbf u
\]
We defiine $\mathbf S$ as:
\renewcommand\arraystretch{1.5}
\[
\mathbf S=-\mathbf u \cdot \nabla \mathbf u-\Pi=
\begin{bmatrix}
-u\frac{\de u}{\de x}-v\frac{\de u}{\de y}-w\frac{\de u}{\de z}-\frac{\de \bar p}{\de x}\\
-u\frac{\de v}{\de x}-v\frac{\de v}{\de y}-w\frac{\de v}{\de z}-\frac{\de \bar p}{\de y}\\
-u\frac{\de w}{\de x}-v\frac{\de w}{\de y}-w\frac{\de w}{\de z}-\frac{\de \bar p}{\de z}\\
\end{bmatrix}
\]
\renewcommand\arraystretch{1}
The velocity space derivatives can be reworked in this way (we show the procedure just for the terms of the first row of $\mathbf S$, the other ones are analogous):
\renewcommand\arraystretch{1.5}
\[
\begin{array}{lcl}
\frac{\de (uu)}{\de x}=2u\frac{\de u}{\de x} & \Rightarrow & u\frac{\de u}{\de x}=\frac{\de (uu)}{\de x} -u\frac{\de u}{\de x}\\
\frac{\de (uv)}{\de y}=v\frac{\de u}{\de y}+u\frac{\de v}{\de y} & \Rightarrow & v\frac{\de u}{\de y}=\frac{\de (uv)}{\de y} -u\frac{\de v}{\de y}\\
\frac{\de (uw)}{\de z}=w\frac{\de u}{\de z}+u\frac{\de w}{\de z} & \Rightarrow & w\frac{\de u}{\de z}=\frac{\de (uw)}{\de z} -u\frac{\de w}{\de z}\\
\end{array}
\]
\renewcommand\arraystretch{1}
So the $\mathbf S$ term can be rewritten as (in the last part the hypthesis of uncompressible fluid has been settled and the value of the mean pressure gradient has been substituted):
\renewcommand\arraystretch{1.5}
\[
\mathbf S=
\begin{bmatrix}
-\frac{\de (uu)}{\de x}-\frac{\de (uv)}{\de y}-\frac{\de (uw)}{\de z}+u(\nabla\cdot \mathbf u)-\frac{\de \bar p}{\de x}\\
-\frac{\de (uv)}{\de x}-\frac{\de (vv)}{\de y}-\frac{\de (vw)}{\de z}+v(\nabla\cdot \mathbf u)-\frac{\de \bar p}{\de y}\\
-\frac{\de (uw)}{\de x}-\frac{\de (vw)}{\de y}-w\frac{\de (ww)}{\de z}+w(\nabla\cdot \mathbf u)-\frac{\de \bar p}{\de z}\\
\end{bmatrix}
=
\begin{bmatrix}
-\frac{\de (uu)}{\de x}-\frac{\de (uv)}{\de y}-\frac{\de (uw)}{\de z}+1\\
-\frac{\de (uv)}{\de x}-\frac{\de (vv)}{\de y}-\frac{\de (vw)}{\de z}\\
-\frac{\de (uw)}{\de x}-\frac{\de (vw)}{\de y}-w\frac{\de (ww)}{\de z}\\
\end{bmatrix}
\]
\renewcommand\arraystretch{1}
Then we rewrite the Navier--Stokes equation as it is used in the code:
\begin{equation}
\frac{\de \mathbf u}{\de t}=\mathbf S-\nabla p'+\frac{1}{\Re_\tau}\Nabla^2 \mathbf u
\end{equation}

\section{$\curl$(N--S)}
We take the curl of the Navier--Stokes equation:
\[
\curl\left(\frac{\de \mathbf u}{\de t}\right)=\curl\mathbf S-\curl\left(\nabla p'\right)+\curl\left(\frac{1}{\Re_\tau}\Nabla^2 \mathbf u\right)
\]
In our case the curl operator and the time and/or space derivatives can be switched, since $\mathbf u$ verifies the hypothesis of the Schwarz theorem\footnote{For the Schwarz theorem the order of partial derivatives can be interchanged (under certain conditions).}.\\
Thus we recall the definition of the vorticity $\boldsymbol \omega$:
\[
\boldsymbol \omega=\curl\mathbf u
\]
\[
\frac{\de (\curl \mathbf u)}{\de t}=\curl\mathbf S-\curl\left(\nabla p'\right)+\frac{1}{\Re_\tau}\Nabla^2\left(\curl \mathbf u\right)
\]
The curl of the gradient of a scalar field is zero (it can be easily proved when the Schwarz theorem holds), so the curl of the fluctuating pressure gradient is zero.
\begin{equation}
\frac{\de \boldsymbol \omega}{\de t}=\curl\mathbf S+\frac{1}{\Re_\tau}\Nabla^2\boldsymbol \omega
\label{eq: vor2}
\end{equation}
The above equation is a second order equation for the vorticity $\boldsymbol \omega$.

\section{$\curl(\curl$N--S$)$}
At this point we take twice the curl of the Navier--Stokes equation:
\[
\curl\left(\curl\frac{\de \mathbf u}{\de t}\right)=\curl(\curl\mathbf S)-\curl\left(\curl\nabla p'\right)+\frac{1}{\Re_\tau}\curl\left(\curl\Nabla^2 \mathbf u\right)
\]
First of all we recall an useful vectorial identity:
\[
\curl(\curl \mathbf A)=\nabla(\nabla\cdot\mathbf A)-\Nabla^2\mathbf A
\]
As already stated the curl of the fluctuating pressure gradient is always zero, so we have:
\[
\nabla\left(\nabla\cdot\frac{\de \mathbf u}{\de t}\right)-\Nabla^2\frac{\de \mathbf u}{\de t}=\nabla(\nabla\cdot\mathbf S)-\Nabla^2\mathbf S+\frac{1}{\Re_\tau}\left[\nabla(\nabla\cdot\Nabla^2 \mathbf u)-\Nabla^2\Nabla^2 \mathbf u\right]
\]
\renewcommand\arraystretch{1.5}
\[
\begin{split}
\nabla\cdot\Nabla^2 \mathbf u&=\nabla \cdot
\begin{bmatrix}
\frac{\de^2 u}{\de x^2}+\frac{\de^2 u}{\de y^2}+\frac{\de^2 u}{\de z^2}\\
\frac{\de^2 v}{\de x^2}+\frac{\de^2 v}{\de y^2}+\frac{\de^2 v}{\de z^2}\\
\frac{\de^2 w}{\de x^2}+\frac{\de^2 w}{\de y^2}+\frac{\de^2 w}{\de z^2}\\
\end{bmatrix} =\\
&=\frac{\de}{\de x}\left( \frac{\de^2 u}{\de x^2}+\frac{\de^2 u}{\de y^2}+\frac{\de^2 u}{\de z^2} \right) +
\frac{\de}{\de y} \left( \frac{\de^2 v}{\de x^2}+\frac{\de^2 v}{\de y^2}+\frac{\de^2 v}{\de z^2} \right)+
\frac{\de}{\de z} \left( \frac{\de^2 w}{\de x^2}+\frac{\de^2 w}{\de y^2}+\frac{\de^2 w}{\de z^2} \right)=\\
&=\frac{\de^2}{\de x^2}\left( \frac{\de u}{\de x}+\frac{\de v}{\de y}+\frac{\de w}{\de z} \right)+
\frac{\de^2}{\de y^2}\left( \frac{\de u}{\de x}+\frac{\de v}{\de y}+\frac{\de w}{\de z} \right)+
\frac{\de^2}{\de z^2}\left( \frac{\de u}{\de x}+\frac{\de v}{\de y}+\frac{\de w}{\de z} \right)=\\
&=\nabla^2 \nabla\cdot\mathbf u
\end{split}
\]
\renewcommand\arraystretch{1}
The above term is zero for an uncompressible flow.\\
\renewcommand\arraystretch{1.5}
\[
\Nabla^2\Nabla^2\mathbf u=\Nabla^4 \mathbf u=
\begin{bmatrix}
\frac{\de^4 u}{\de x^4}+\frac{\de^4 u}{\de y^4}+\frac{\de^4 u}{\de z^4}+2\left( \frac{\de^4 u}{\de x^2\de y^2} + \frac{\de^4 u}{\de x^2\de z^2} +\frac{\de^4 u}{\de y^2\de z^2}\right)\\
\frac{\de^4 v}{\de x^4}+\frac{\de^4 v}{\de y^4}+\frac{\de^4 v}{\de z^4}+2\left( \frac{\de^4 v}{\de x^2\de y^2} + \frac{\de^4 v}{\de x^2\de z^2} +\frac{\de^4 v}{\de y^2\de z^2}\right)\\
\frac{\de^4 w}{\de x^4}+\frac{\de^4 w}{\de y^4}+\frac{\de^4 w}{\de z^4}+2\left( \frac{\de^4 w}{\de x^2\de y^2} + \frac{\de^4 w}{\de x^2\de z^2} +\frac{\de^4 w}{\de y^2\de z^2}\right)\\
\end{bmatrix}
\]
\renewcommand\arraystretch{1}
For an uncompressible flow the above equation results in a fourth-order equation for the velocity:
\begin{equation}
\frac{\de}{\de t}(\Nabla^2 \mathbf u)=\Nabla^2 \mathbf S -\nabla(\nabla\cdot\mathbf S)+\frac{1}{\Re_\tau}\Nabla^4 \mathbf u
\end{equation}


\section{Vectorial equations}
Here are reported the four equations of the problem:
\begin{equation}
\begin{cases}
\nabla\cdot\mathbf u=0 &\textnormal{continuity equation}\\
\boldsymbol\omega=\curl\mathbf u&\textnormal{vorticity definition}\\
\frac{\de \boldsymbol \omega}{\de t}=\curl\mathbf S+\frac{1}{\Re_\tau}\Nabla^2\boldsymbol \omega&\textnormal{2$^{nd}$ order equation for vorticity}\\
\frac{\de}{\de t}(\Nabla^2 \mathbf u)=\Nabla^2 \mathbf S -\nabla(\nabla\cdot\mathbf S)+\frac{1}{\Re_\tau}\Nabla^4 \mathbf u&\textnormal{4$^{nd}$ order equation for velocity}\\
\end{cases}
\end{equation}
The code solves the transport of velocity and vorticity only for the $z$ component and evaluates $u$ and $v$ using the continuity equation (scalar) and the third component of the vorticity definition $\omega_z$ (Eq. \ref{eq: scalarsystem}).
\begin{equation}
\begin{cases}
\nabla\cdot\mathbf u=0 \\
\omega_z=\frac{\de v}{\de x}-\frac{\de u}{\de y}\\
\frac{\de \omega_z}{\de t}= \frac{\de S_2}{\de x}-\frac{\de S_1}{\de y}+\frac{1}{\Re_\tau}\left(\frac{\de^2 \omega_z}{\de x^2}+\frac{\de^2\omega_z}{\de y^2}+\frac{\de^2\omega_z}{\de z^2} \right) \\
\begin{split}
\frac{\de}{\de t}&\left( \frac{\de^2w}{\de x^2}+\frac{\de^2w}{\de y^2}+\frac{\de^2w}{\de z^2}  \right)=\frac{\de^2S_3}{\de x^2}+\frac{\de^2S_3}{\de y^2}+\frac{\de^2S_3}{\de z^2}-\left( \frac{\de^2 S_1 }{\de x \de z}+\frac{\de^2 S_2 }{\de y \de z} +\frac{\de^2 S_3 }{\de z^2} \right) +\\
&+\frac{1}{\Re_\tau}\left[ \frac{\de^4 w}{\de x^4}+\frac{\de^4 w}{\de y^4}+\frac{\de^4 w}{\de z^4}+2\left( \frac{\de^4 w}{\de x^2\de y^2} + \frac{\de^4 w}{\de x^2\de z^2} +\frac{\de^4 w}{\de y^2\de z^2}\right) \right]  \\
\end{split}
\end{cases}
\label{eq: scalarsystem}
\end{equation}

\section{Spatial discretization}
$x$ and $y$ are periodical direction (in these directions are enforced periodic boundary conditions), so they can be discretized using Fourier series. $z$ direction is not periodical; in this direction Chebyshev polynomials are employed.\\
\renewcommand\arraystretch{1.5}
\[
\begin{array}{lccl}
x_i=(i-1)\dfrac{L_x}{N_x-1} & & i=1,...,N_x\\
x_j=(j-1)\dfrac{L_x}{N_y-1} & & j=1,...,N_y\\
z_k=\cos\left(\frac{k-1}{N_z-1}\right) && k=1,...,N_z\\
\end{array}
\]
\renewcommand\arraystretch{1}
We then define the wavenumbers $k_1$ and $k_2$ of the Fourier series:
\[
k_1=\frac{2\pi n_1}{L_x}
\]
\[
k_2=\frac{2\pi n_2}{L_y}
\]
The Fourier transform of a general quantity (in this case one of the components of the velocity) is:
\[
u_k(x,y,z,t)=\sum_{n_1=N_1/2+1}^{N_1/2}\sum_{n_2=N_2/2+1}^{N_2/2}\sum_{n_3=0}^{N_3-1}\hat u_i(k_1,k_2,n_3,t)T_{n_3}(z)e^{i(k_1x+k_2y)}
\]
The \string^ symbol denotes the Fourier coefficients, while $T_k(z)$ is the $k^{th}$ Chebyshev polynomial. It can be seen that derivative can be easily taken in Fourier space.\\
The discretized continuity equation results in:
\begin{equation}
\begin{split}
\sum_{n_1}\sum_{n_2}\sum_{n_3}\left[ \left(\vphantom{\frac{\de T_{n_3}(z)}{\de z}}\hat u(k_1,k_2,n_3,t)T_{n_3}(z)ik_1+\hat v(k_1,k_2,n_3,t)T_{n_3}(z)ik_2+\right .\right . \\
\left .\left . +\hat w(k_1,k_2,n_3,t)\frac{\de T_{n_3}(z)}{\de z}\right) e^{i(k_1x+k_2y)} \right]=0
\end{split}
\end{equation}
Similarly the vorticity definition is:
\begin{equation}
\begin{split}
\sum_{n_1}\sum_{n_2}\sum_{n_3}\left(\hat\omega_z(k_1,k_2,n_3,t)T_{n_3}(z)e^{i(k_1x+k_2y)}  \right)=\\
=\sum_{n_1}\sum_{n_2}\sum_{n_3}\left[\Big(\hat v(k_1,k_2,n_3,t)ik_1-\hat u(k_1,k_2,n_3,t)ik_2\Big)T_{n_3}(z)e^{i(k_1x+k_2y)}  \right]
\end{split}
\end{equation}
The second order vorticity equation is:
\[
\begin{split}
\frac{\de}{\de t}\sum_{n_1}\sum_{n_2}\sum_{n_3}\left(\hat\omega_z(k_1,k_2,n_3,t)T_{n_3}(z)e^{i(k_1x+k_2y)}  \right)=\\
=\sum_{n_1}\sum_{n_2}\sum_{n_3}\left[\left(\hat S_2(k_1,k_2,n_3,t)ik_1-\hat S_1(k_1,k_2,n_3,t)ik_2\right)T_{n_3}(z)e^{i(k_1x+k_2y)}  \right]+\\
+\frac{1}{\Re_\tau}\sum_{n_1}\sum_{n_2}\sum_{n_3}\left[\left(-k_1^2T_{n_3}(z)-k_2^2T_{n_3}(z)+\frac{\de^2 T_{n_3}(z)}{\de z^2} \right)\hat\omega_z(k_1,k_2,n_3,t)e^{i(k_1x+k_2y)}  \right]
\end{split}
\]
We define $k^2=k_1^2+k_2^2$.
\begin{equation}
\begin{split}
\frac{\de}{\de t}\sum_{n_1}\sum_{n_2}\sum_{n_3}\left(\hat\omega_z(k_1,k_2,n_3,t)T_{n_3}(z)e^{i(k_1x+k_2y)}  \right)=\\
=\sum_{n_1}\sum_{n_2}\sum_{n_3}\left[\left(\hat S_2(k_1,k_2,n_3,t)ik_1-\hat S_1(k_1,k_2,n_3,t)ik_2\right)T_{n_3}(z)e^{i(k_1x+k_2y)}  \right]+\\
+\frac{1}{\Re_\tau}\sum_{n_1}\sum_{n_2}\sum_{n_3}\left[\left(-k^2T_{n_3}(z)+\frac{\de^2 T_{n_3}(z)}{\de z^2} \right)\hat\omega_z(k_1,k_2,n_3,t)e^{i(k_1x+k_2y)}  \right]
\end{split}
\end{equation}
The fourth order equation for velocity is:
\[
\begin{split}
\frac{\de}{\de t}\sum_{n_1}\sum_{n_2}\sum_{n_3}\left[\left(-k_1^2T_{n_3}(z)-k_2^2T_{n_3}(z)+\frac{\de T_{n_3}(z)}{\de z}\right)\hat w(k_1,k_2,n_3,t)e^{i(k_1x+k_2y)}  \right]=\\
\sum_{n_1}\sum_{n_2}\sum_{n_3}\left[\left(-k_1^2T_{n_3}(z)-k_2^2T_{n_3}(z)+\frac{\de^2 T_{n_3}(z)}{\de z^2}\right)\hat S_3(k_1,k_2,n_3,t)e^{i(k_1x+k_2y)}  \right]-\\
-\sum_{n_1}\sum_{n_2}\sum_{n_3}\left[\left(ik_1\frac{\de T_{n_3}(z)}{\de z}\hat S_1(k_1,k_2,n_3,t)+ik_2\frac{\de T_{n_3}(z)}{\de z}\hat S_2(k_1,k_2,n_3,t)+ \right.\right.\\
\left.\left.+\frac{\de^2 T_{n_3}(z)}{\de z^2}\hat S_3(k_1,k_2,n_3,t)\right)e^{i(k_1x+k_2y)}  \right]+\\
+\frac{1}{\Re_\tau}\sum_{n_1}\sum_{n_2}\sum_{n_3}\left[\left(k_1^4T_{n_3}(z)+k_2^4 T_{n_3}(z)+ \frac{\de^4 T_{n_3}(z)}{\de z^4}+2\left( k_1^2k_2^2-k_1^2\frac{\de^2 T_{n_3}(z)}{\de z^2}-k_2^2\frac{\de^2 T_{n_3}(z)}{\de z^2}\right)\right)\right. \\
\left.\hat w(k_1,k_2,n_3,t)e^{i(k_1x+k_2y)}  \vphantom{\frac{\de^2 T_{n_3}(z)}{\de z^2}}\right]
\end{split}
\]
Substituting the equation for $k$ we obtain:
\begin{equation}
\begin{split}
\frac{\de}{\de t}\sum_{n_1}\sum_{n_2}\sum_{n_3}\left[\left(-k^2T_{n_3}(z)+\frac{\de T_{n_3}(z)}{\de z} \right)\hat w(k_1,k_2,n_3,t)e^{i(k_1x+k_2y)}  \right]=\\
\sum_{n_1}\sum_{n_2}\sum_{n_3}\left[\left(-k^2T_{n_3}(z)+\frac{\de^2 T_{n_3}(z)}{\de z^2}\right)\hat S_3(k_1,k_2,n_3,t)e^{i(k_1x+k_2y)}  \right]-\\
-\sum_{n_1}\sum_{n_2}\sum_{n_3}\left[\left(ik_1\frac{\de T_{n_3}(z)}{\de z}\hat S_1(k_1,k_2,n_3,t)+ik_2\frac{\de T_{n_3}(z)}{\de z}\hat S_2(k_1,k_2,n_3,t)+ \right.\right.\\
\left.\left.+\frac{\de^2 T_{n_3}(z)}{\de z^2}\hat S_3(k_1,k_2,n_3,t)\right)e^{i(k_1x+k_2y)}  \right]+\\
+\frac{1}{\Re_\tau}\sum_{n_1}\sum_{n_2}\sum_{n_3}\left[\left(k_1^4T_{n_3}(z)+k_2^4 T_{n_3}(z)+ \frac{\de^4 T_{n_3}(z)}{\de z^4}+2\left( k_1^2k_2^2-k^2\frac{\de^2 T_{n_3}(z)}{\de z^2}\right)\right)\right. \\
\left.\hat w(k_1,k_2,n_3,t)e^{i(k_1x+k_2y)}  \vphantom{\frac{\de^2 T_{n_3}(z)}{\de z^2}}\right]
\end{split}
\end{equation}
We now introduce a more compact notation:
\[
\hat q=\sum_{n_1}\sum_{n_2}\sum_{n_3}\hat q(k_1,k_2,n_3,t)T_{n_3}(z)e^{i(k_1x+k_2y)}
\]
\begin{itemize}
\item Continuity equation
\[
ik_1\hat u+ik_2\hat v+\frac{\de \hat w}{\de z}=0
\]
\item Vorticity definition
\[
\hat \omega_z=ik_1\hat v-ik_2\hat u
\]
\item Vorticity equation (2$^{nd}$ order)
\[
\frac{\de\hat\omega_z}{\de t}=ik_1\hat S_2-ik_2\hat S_1+\frac{1}{\Re_\tau}\left(\frac{\de^2 \hat\omega_z}{\de z^2}-k^2\hat\omega_z  \right)
\]
\item Velocity equation ($4^{th}$ order)
\[
\begin{split}
\frac{\de}{\de t}\left(\frac{\de^2\hat w}{\de z^2}-k^2\hat w  \right)=&-k^2\hat S_3-ik_1\frac{\de \hat S_1}{\de z}-ik_2\frac{\de \hat S_2}{\de z}+\\
&+ \frac{1}{\Re_\tau}\left[k_1^4\hat w+k_2^4\hat w+\frac{\de^4 \hat w}{\de z^4}+2\left(k_1^2k_2^2\hat w-k_2\frac{\de^2\hat w}{\de z^2}  \right)\right]
\end{split}
\]
\end{itemize}


\section{Time discretization}
For time discretization two differents algorithms are used: Crank--Nicolson and Adam--Bashforth. The first one is an implicit algorithm and it is used for the diffusive terms; being implicit it favours the method stability. The Adam--Bashforth algorithm instead is used for the convective terms and it is an explicit method. In the following the superscripts denotes the time-step: $n+1$ is the time-step that is going to be evaluated (unknown), $n$ is the current time-step (known) and $n-1$ is the previous time-step (known).

\subsection{Crank--Nicolson}
\[
\frac{u^{n+1}-u^n}{\Delta t}=\frac{F^{n+1}+F^n}{2}
\]

\subsection{Adam--Bashforth}
\[
\frac{u^{n+1}-u^n}{\Delta t}=\frac 3 2 F^n-\frac 1 2 F^{n-1}
\]
Since at the first time-step ($n=0$) only the data for $n=0$ (initial conditions) are available (so there are no data for $n-1$), an explicit Euler algorithm is used:
\[
\frac{u^{n+1}-u^n}{\Delta t}=F^n
\]


\section{Equations solved}
Since the continuity equation and the vorticity definition contain no time derivatives, their discretization is easy, so we will report only the final form without showing all the passages.

\subsection{2$^{nd}$ order equation for vorticity}
We can split the right hand side of this equation in two terms, a diffusive one $\psi$ and a convective one $\xi$.
\[
\begin{array}{l}
\xi=ik_1\hat S_2-ik_2\hat S_2\\[2ex]
\psi=\dfrac{1}{\Re_\tau}\left( \dfrac{\de^2 \hat \omega_z}{\de z^2}-k^2\hat\omega_z\right)
\end{array}
\]
The resulting equation is:
\[
\frac{\de \hat\omega_z}{\de t}=\xi+\psi
\]
Applying the two time discretization algorithms presented before to the previous equation, we obtain:
\[
\frac{\hat\omega_z^{n+1}-\hat\omega_z^n}{\Delta t}=\frac 3 2 \xi^n-\frac 1 2 \xi^{n-1}+\frac{\psi^{n+1}+\psi^n}{2}
\]
Substituting the convective and diffusive terms into the equation we have:
\[
\begin{split}
\hat\omega_z^{n+1}=&\hat\omega_z^n+\Delta t \left[ \frac 3 2 \left( i k_1\hat S_2^n -i k_2\hat S_1^n \right)-\frac{1}{2}\left( ik_1\hat S_2^{n-1}-ik_2\hat S_1^{n-1}  \right) \right]+\\
&+\frac{\Delta t}{2\Re_\tau}\left( \frac{\de^2 \hat\omega_z^{n+1}}{\de z^2}-k^2\hat\omega_z^{n+1} +\frac{\de^2 \hat\omega_z^{n}}{\de z^2}-k^2\hat\omega_z^{n} \right)
\end{split}
\]
At this point we substitute the definition of vorticity in $\hat\omega_z^n$:
\[
\hat\omega_z^n=ik_1\hat v^n-ik_2\hat u^n
\]
Rearranging all teh terms and defining the historical term $H_i^n$ as:
\[
H_i^n=\Delta t\left[\frac 3 2 \hat S_i^n-\frac 12 \hat S_i^{n-1}+\frac{1}{2\Re_\tau}\frac{\de^2 \hat u_i^n}{\de z^2}+\left(\frac{1}{\Delta t}-\frac{k^2}{2\Re_\tau} \right)\hat u_i^n  \right]
\]
we obtain:
\begin{equation}
\left(\frac{k^2\Delta t}{2\Re_\tau}+1 \right)\hat\omega_z^{n+1}-\frac{\Delta t}{2\Re_\tau}\frac{\de^2 \hat\omega_z^{n+1}}{\de z^2}=ik_1H_2^n-ik_2H_1^n
\end{equation}

\subsection{4$^{th}$ order equation for velocity}
In the same way as we did for the vorticity transport equation, we can split this equation in a diffusive term and in a convective one.
\[
\begin{array}{l}
\xi=-k^2 \hat S_3 -ik_1\dfrac{\de\hat S_1}{\de z}-ik_2\dfrac{\de\hat S_2}{\de z} \\[2ex]
\psi=\dfrac{1}{\Re_\tau}\left[ k_1^4\hat w +k_2^4\hat w+2\left(k_1^2k_2^2\hat w-k^2\dfrac{\de^2 \hat w}{\de z^2}  \right) \right]
\end{array}
\]
This way we obtain the follwing equation:
\[
\frac{\de}{\de t}\left(\frac{\de^2\hat w}{\de z^2}-k^2\hat w\right)=\xi+\psi
\]
We discretize this equation as seen before:
\[
\frac{1}{\Delta t}\left(\frac{\de^2\hat w^{n+1}}{\de z^2}-k^2\hat w^{n+1}-\frac{\de^2\hat w^n}{\de z^2}+k^2\hat w^n\right)=\frac 3 2 \xi^n-\frac 1 2 \xi^{n-1}+\frac{\psi^{n+1}+\psi^n}{2}
\]
\[
\frac{\de^2\hat w^{n+1}}{\de z^2}-k^2\hat w^{n+1}-\frac{\Delta t}{2}\psi^{n+1}=\frac{\de^2\hat w^n}{\de z^2}-k^2\hat w^n+\Delta t\left( \frac 3 2 \xi^n-\frac 1 2 \xi^{n-1}+\frac{\psi^n}{2}\right)
\]
Now we substitute both the equation for the convective and the diffusive term in the equation:
\[
\begin{split}
-\frac{\Delta t}{2\Re_\tau}\frac{\de^4\hat w^{n+1}}{\de z^4}+\left(1+\frac{\Delta t}{\Re_\tau}k^2 \right)\frac{\de^2 \hat w^{n+1}}{\de z^2}+ \left(-k^2-\frac{\Delta t}{2\Re_\tau}(k_1^4+k_2^4)-\frac{\Delta t}{\Re_\tau}k_1^2k_2^2  \right)\hat w^{n+1}=\\
=\frac{\de^2 \hat w^n}{\de z^2}-k^2\hat w^n+\Delta t \left[ \frac 32 \left(-k^2\hat S_3^{n}-ik_1\frac{\de \hat S_1^{n}}{\de z}-ik_2\frac{\de \hat S_2^{n}}{\de z}  \right)- \right .\\
\left.  -\frac 1 2 \left(-k^2\hat S_3^{n-1}-ik_1\frac{\de \hat S_1^{n-1}}{\de z}-ik_2\frac{\de \hat S_2^{n-1}}{\de z}  \right)+\right . \\
\left.+\frac{1}{2\Re_\tau}\left(k_1^4\hat w^n+k_2^4\hat w^n+\frac{\de^4\hat w^n}{\de z^4}+2\left( k_1^2k_2^2\hat w^n -k^2\frac{\de^2 \hat w^n}{\de z^2} \right)  \right)  \right]
\end{split}
\]
Rearranging the right-hand side of the equation we obtain:
\[
\begin{split}
-\frac{\Delta t}{2\Re_\tau}\frac{\de^4\hat w^{n+1}}{\de z^4}+\left(1+\frac{\Delta t}{\Re_\tau}k^2 \right)\frac{\de^2 \hat w^{n+1}}{\de z^2}+ \left(-k^2-\frac{\Delta t}{2\Re_\tau}(k_1^4+k_2^4)-\frac{\Delta t}{\Re_\tau}k_1^2k_2^2  \right)\hat w^{n+1}=\\
=\Delta t \left[\frac 32 \left(-ik_1\frac{\de \hat S_1^n}{\de z}\right)+\frac1 2 ik_1\frac{\de \hat S_1^{n-1}}{\de z} \right] +\Delta t \left[\frac 32 \left(-ik_2\frac{\de \hat S_2^n}{\de z}\right)+\frac1 2 ik_2\frac{\de \hat S_2^{n-1}}{\de z} \right]+\\
+\Delta t \left[ -\frac 32 k^2\hat S_3^n+\frac 1 2 k^2 \hat S_3^{n-1}+ \frac{1}{2\Re_\tau}\left(k_1^4\hat w^n+k_2^4\hat w^n+\frac{\de^4\hat w^n}{\de z^4}+2 k_1^2k_2^2\hat w^n -2k^2\frac{\de^2 \hat w^n}{\de z^2}  \right)  \right]
\end{split}
\]
Now we recall the definition of the histry term $H_i^n$ and we differentiate it in $z$:
\[
\frac{\de H_i^n}{\de z}=\left(\frac 3 2\frac{\de \hat S_i^n}{\de z}-\frac 1 2 \frac{\de \hat S_i^{n-1}}{\de z}\right)\Delta t+\frac{\Delta t}{2\Re_\tau}\frac{\de^3\hat u_i^n}{\de z^3}+\left(1-\frac{\Delta t k^2}{2\Re_\tau}\right)\frac{\de \hat i_i^n}{\de z}
\]
Substituting this result in the 4$^{th}$ order equation for velocity and rearranging the terms, we have:
\[
\begin{split}
-\frac{\Delta t}{2\Re_\tau}\frac{\de^4\hat w^{n+1}}{\de z^4}+\left(1+\frac{\Delta t}{\Re_\tau}k^2 \right)\frac{\de^2 \hat w^{n+1}}{\de z^2}+ \left(-k^2-\frac{\Delta t}{2\Re_\tau}(k_1^4+k_2^4)-\frac{\Delta t}{\Re_\tau}k_1^2k_2^2  \right)\hat w^{n+1}=\\
=-ik_1\frac{\de H_1^n}{\de z}-ik_2\frac{\de H_2^n}{\de z}+\frac{\Delta t}{2\Re_\tau}\frac{\de^3}{\de z^3}\left(ik_1\hat u^n+ik_2\hat v^n+\frac{\de\hat w^n}{\de z}  \right)+\\
+\left(1-\frac{\Delta t k^2}{2\Re_\tau}\right)\frac{\de}{\de z}\left(ik_1\hat u^n+ik_2\hat v^n+\frac{\de\hat w^n}{\de z}  \right)-k^2H_3^n-\frac{\Delta t k^4}{2\Re_\tau}\hat w^n+\\
+\frac{\Delta t}{2\Re_\tau}(k_1^4+k_2^4+k_1^2k_2^2)\hat w^n
\end{split}
\]
In this equation appears twice the continuity equation for an uncompressible flow, so it can be simplified in:
\[
\begin{split}
-\frac{\Delta t}{2\Re_\tau}\frac{\de^4\hat w^{n+1}}{\de z^4}+\left(1+\frac{\Delta t}{\Re_\tau}k^2 \right)\frac{\de^2 \hat w^{n+1}}{\de z^2}+ \left(-k^2-\frac{\Delta t}{2\Re_\tau}(k_1^4+k_2^4)-\frac{\Delta t}{\Re_\tau}k_1^2k_2^2  \right)\hat w^{n+1}=\\
=-ik_1\frac{\de H_1^n}{\de z}-ik_2\frac{\de H_2^n}{\de z}-k^2H_3^n
\end{split}
\]
We now define a second history term $H^n$:
\[
H^n=ik_1\frac{\de H_1^n}{\de z}+ik_2\frac{\de H_2^n}{\de z}+k^2H_3^n
\]
This last equation has been checked and verified according to \cite{Guasti_2012}.
\begin{equation}
\frac{\Delta t}{2\Re_\tau}\frac{\de^4\hat w^{n+1}}{\de z^4}-\left(1+\frac{\Delta t}{\Re_\tau}k^2 \right)\frac{\de^2 \hat w^{n+1}}{\de z^2}+ k^2\left(1+\frac{\Delta t k^2}{2\Re_\tau} \right)\hat w^{n+1}=H^n
\end{equation}



\subsection{Discretized system of equations}
Here is reported the discretized system of equations solved from the code.
\begin{equation}
\begin{cases}
ik_1\hat u^{n+1}+ik_2\hat v^{n+1}+\dfrac{\de \hat w^{n+1}}{\de z}=0\\[2ex]
\hat \omega_z^{n+1}=ik_1\hat v^{n+1}-ik_2\hat u^{n+1}\\[2ex]
\left(\dfrac{k^2\Delta t}{2\Re_\tau}+1 \right)\hat\omega_z^{n+1}-\dfrac{\Delta t}{2\Re_\tau}\dfrac{\de^2 \hat\omega_z^{n+1}}{\de z^2}=ik_1H_2^n-ik_2H_1^n\\[4ex]
\dfrac{\Delta t}{2\Re_\tau}\dfrac{\de^4\hat w^{n+1}}{\de z^4}-\left(1+\dfrac{\Delta t}{\Re_\tau}k^2 \right)\dfrac{\de^2 \hat w^{n+1}}{\de z^2}+ k^2\left(1+\dfrac{\Delta t k^2}{2\Re_\tau} \right)\hat w^{n+1}=H^n
\end{cases}
\end{equation}
The continuity equation and the vorticity definition are solved after the 2$^{nd}$ order equation for vorticity and the 4$^{th}$ order equation for velocity, so once $\hat w^{n+1}$ and $\hat \omega_z^{n+1}$ are known.













